\chapter*{Resumo}
\addcontentsline{toc}{chapter}{Resumo}

O modelo de volatilidade estocástica \citep{kim} constitui uma classe de modelos bastante importante, pois visa ajustar séries de dados temporais cuja variabilidade é aleatória. Tal modelo tem sido abordado tanto do ponto de vista clássico, quanto Bayesiano. Porém, várias de suas especificações ainda carecem de uma solução mais adequada.

O recente trabalho de \cite{kastner}  apresentou desenvolvimentos importantes quanto ao método Bayesiano de estimação dos parâmetros do modelo. Os autores apresentam uma ideia bastante inovadora denominada, em inglês, de \textit{Ancillarity-Sufficiency Interweaving Strategy}, que consiste em alternar as parametrizações do modelo durante o processo de estimação. Entretanto, a proposta de estimação da variável latente, que rege o modelo, é um pouco complicada. Com o presente trabalho de dissertação propõe-se uma forma alternativa, e mais simples, de se estimar a variável latente do processo baseada no trabalho de \cite{mccormick}.

Nesta dissertação, foram adaptadas as metodologias propostas pelos dois últimos autores de modo a sugerir uma forma alternativa de estimar os parâmetros do modelo de volatilidade estocástica. Utilizando dados simulados, foi avaliada a efetividade da nova proposta. Ao final desse trabalho foram destacados os problemas emergentes do procedimento proposto e sugeridas algumas novas alternativas para resolve-los.
\newline

\noindent \textbf{Palavras Chave}: \textit{Estatística Bayesiana, Modelo de Volatilidade Estocástica, Modelos Lineares Dinâmicos, Ancillarity-Sufficiency Interweaving Strategy, JAGS}
